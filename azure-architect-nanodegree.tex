\documentclass[a4paper,14pt]{report}
\usepackage{listings}
\usepackage{color}
\pagestyle{headings}
\usepackage[utf8]{inputenc}
\author{Hugh McKenna}

\definecolor{dkgreen}{rgb}{0,0.6,0}
\definecolor{gray}{rgb}{0.5,0.5,0.5}
\definecolor{white}{rgb}{1,1,1}
\definecolor{mauve}{rgb}{0.58,0,0.82}

\lstset{frame=tb,
    language=bash,
    aboveskip=3mm,
    belowskip=3mm,
    showstringspaces=false,
    columns=flexible,
    basicstyle={\small\ttfamily},
    numbers=none,
    color=\color{white},
    numberstyle=\tiny\color{gray},
    keywordstyle=\color{blue},
    commentstyle=\color{dkgreen},
    stringstyle=\color{mauve},
    breaklines=true,
    breakatwhitespace=true,
    tabsize=3
}

\begin{document}
    \chapter{Designing Infrastructure}


    \section{Azure Cloud Capabilities}

    \paragraph{Creating a web server}
    Web servers will be used throughout the course. Simply providing a cloud computer instead of a
    physical machine, running IIS to serve a web application and remote desktop to access and edit.

    \begin{lstlisting}[
        caption = {Creating a VM with IIS Exercise}]
    > az account set --subscription {{ Subscription Name }}
    > az group create --name {{ Resource group name }} --location {{ e.g westeurope }}
    > az az vm create --resource-group azure-cloud-capabilities --name web01 --image Win2019Datacenter --admin-username web01 --public-ip-sku Standard
    > az vm open-port --port 80,443 --resource-group {{ Resource group name }} --name web01
    \end{lstlisting}

    \\
    You will be asked to create a password for the admin user. Once the VM is created you should
    be able to enter the generated IP in a Remote Desktop app and enter the admin password.
    \begin{enumerate}
        \item Visit \%SystemDrive\%/inetpub/wwwroot
        \item Create an index.html with any content you like
        \item Visit the VM using it's public IP
    \end{enumerate}

    \paragraph{High Availability}
    Ensure users are always able to access your services in an optimal state. Whether that is providing fallbacks in the
    case of maintenance or unplanned outage, resources in a localised data center or load balancing between multiple instances.
    \\
    \begin{itemize}
        \item \textbf{Availability Zones} An instance across multiple data centres, within the same region
        \item \textbf{VM Scale Sets} Instances spin up and down based on demand
        \item \textbf{Availablity Sets} Contains multiple Fault domains for unplanned outages, Update domains for planned
        \item \textbf{Load Balancer} Resources in different regions to mitigate against disasters
        \begin{itemize}
            \item \textbf{Application Gateway} is more advanced, can handle routing based on HTTP metadata
            \item \textbf{Front Door} handles encryption requests as well as decryption
            \item \textbf{Azure Load Balancer} balances traffic based on load
            \item \textbf{Traffic Manager} balances traffic based on geolocation
        \end{itemize}
    \end{itemize}
    \begin{lstlisting}[
        caption = {Create 2 VMs and load balancer}]
        > az group create --name azure-cloud-capabilities --location westeurope
        > az vm availability-set create -n azure-cloud-capabilities-avset -g azure-cloud-capabilities --platform-fault-domain-count 2 --platform-update-domain-count 2
        > az network public-ip create -g azure-cloud-capabilities --name webip --allocation-method Static
        > az vm create --resource-group azure-cloud-capabilities --name web01 --image Win2019Datacenter --admin-username web01 --availability-set azure-cloud-capabilities-avset
        > az vm create --resource-group azure-cloud-capabilities --name web02 --image Win2019Datacenter --admin-username web02 --availability-set azure-cloud-capabilities-avset
        > az vm open-port --port 80,443 --resource-group azure-cloud-capabilities --name web01
        > az vm open-port --port 80,443 --resource-group azure-cloud-capabilities --name web02
        > az network lb create -g azure-cloud-capabilities -n weblb --frontend-ip-name frontendpool --backend-pool-name backendpool --public-ip-address webip
        > az network vnet create -g azure-cloud-capabilities -n webvnet --subnet-name websubnet
        > az network lb probe create -g azure-cloud-capabilities --lb-name weblb -n weblbprobe --protocol tcp --port 80
        > az network lb rule create -g azure-cloud-capabilities --lb-name weblb -n weblbrule --protocol tcp --frontend-port 80 --backend-port 80 --frontend-ip-name frontendpool --backend-pool-name backendpool --probe-name weblbprobe
    \end{lstlisting}

    \paragraph{Backend Pools} The VM will automatically be assigned an IP. This means it can't be put behind in a load balancer
    in a back end pool. You can manually disassociate the IP in the Azure GUI, but there's probably a better way. As it stands
    using the portal GUI is the easiest way to add your VMs to the load balancer pool.
    \begin{lstlisting}[caption={To Delete All Resources In The Group}]
    az group delete --no-wait -n azure-cloud-capabilities
    \end{lstlisting}

    \begin{lstlisting}[caption={DNN Web App}]
        > az mysql server create -n mydnn -g dnn-web-app -l westeurope --geo-redundant-backup Disabled --sku-name GP_Gen5_2 --storage-size 256000 -u mydnn -p {{ password }}
        > az appservice plan create -g dnn-web-app -n dnnwebappplan
        > az webapp create -g dnn-web-app -n dnnwebapp --runtime dotnet:6 --plan dnnwebappplan
    \end{lstlisting}
    \\
    Connect to the web app via FTP and upload the release candidate of the DNN app in the web root.
    \\
    https://github.com/dnnsoftware/Dnn.Platform/releases

\end{document}